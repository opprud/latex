%
\documentclass[journal]{IEEEtran}
\usepackage{graphicx}
\usepackage{placeins}
%\usepackage{caption}
\graphicspath{{./figures/}}
%for drawing figures
\usepackage{tikz}
\usetikzlibrary{shapes,arrows}
\usepackage{verbatim}


%\input{preamble}



% *** PDF, URL AND HYPERLINK PACKAGES ***
%
%\usepackage{url}
% url.sty was written by Donald Arseneau. It provides better support for
% handling and breaking URLs. url.sty is already installed on most LaTeX
% systems. The latest version can be obtained at:
% http://www.ctan.org/tex-archive/macros/latex/contrib/misc/
% Read the url.sty source comments for usage information. Basically,
% \url{my_url_here}.







% correct bad hyphenation here
\hyphenation{op-tical net-works semi-conduc-tor}


\begin{document}
%
% paper title
% can use linebreaks \\ within to get better formatting as desired
\title{Assessing current Danish initiatives of energy-awareness campaigns }


% author names and affiliations
% use a multiple column layout for up to three different
% affiliations
\author{\IEEEauthorblockN{Morten Opprud Jakobsen}
\IEEEauthorblockA{AUhe\\
Aarhus University Herning\\
Business and social sciences\\
Birk Centerpark 15 - 7400 Herning\\
Email: morten@hih.au.dk \\
Herning, June 3 - 2012}}





% use for special paper notices
%\IEEEspecialpapernotice{(Invited Paper)}




% make the title area
\maketitle


\begin{abstract}
%\boldmath
Danish energy users are often perceived as "green" and pro-renewable. The Danish government has recently presented a new ambitious 2020 Energy-agreement, that aims at large scale initiatives both in terms of energy savings, and conversion of traditional power-plants to renewable energy sources.
Responsibility for saving initiatives, and energy-awareness campaigns, in the years to come, are to be initiated by the utility companies, rather then the ministries, that has led campaigns previously.

This article assesses some of the current and past campaigns, with an aim of indicating what instruments have been used to reach energy consumers, and present saving potentials. This the aim is to get an indication of what effective instruments can be utilized in future IT-aided tools, for enforcing greener habits.

The outcome indicates that...

\textit{Keywords - awareness, greener habits, public energy campaigns}
\end{abstract}

 

\section{Introduction}
% no \IEEEPARstart
Electricity has become a necessity in all developed countries, not only in Denmark on which this article focuses. Yet with fluctuating production prices, due to a growing share being produced by renewable sources, and a long term trend of an annual increase in electricity prices of approximately 5\% \cite{udv_elpriser}, still private households accounts for 20\% of the total electricity consumption in Denmark \cite{energistat}.

The Danish governments recent ambitious goal, of positioning Denmark as a pioneering nation, that must be self-sustainable, having the entire energy consumption covered by renewable energysources by 2050 \cite{energipolitik_2020} requires much of the electricity infrastructure to be rethought, along with many of the usage patterns and habits in terms of electricity usage.

\textit{Energy available on demand} has become a truism, in the Danish society. The fluctuating availability of energy production from renewable energy sources, as wind-turbines, and photovoltaics, will present massive challenges, in terms of large-scale  integration into a future \textit{always available} energy infra-structure. In order to support the Danish governments ambitious goals regarding renewable energy sources, this will require massive innovation to emerge, and probably also accept from the energy users, that \textit{Energy available on demand} might move towards \textit{Using most energy when it is cheap}   

The long term increase in disposable income increases to leverage the availability of electrical appliances in Danish households %\cite{keylist}. 
The increase in energy consuming appliances can be split into both leisure-usage products, such as flat screen TV's gaming consoles, laptops etc. labeled \textit{wants}, and on the other hand white goods and other home appliances, aiding the busy family in their daily business, labeled \textit{needs}.

Stimulating and enforcing new habits in household-appliances like an oven or cooker (\textit{needs}), may enforce energy saving behavior. 
Visualizing actual usage and cost associated with \textit{standby-consumption} and \textit{leave-it-on-behavior}, and perhaps even awarding \textit{good habits}, when using TV's, Laptops etc. can probably affect a family's energy usage, and green focus in a positive way.

It is believed that visualizing a clear distinction between \textit{necessary} and \textit{leisure} usage of energy (\textit{needs} and \textit{wants}) may actually stimulate awareness and interest in the actual nature of the energy usage, and hereby generate a greater awareness and reduce the energy consumption.

The outcome of this article will try to assess relevant initiatives and campaigns,and their impact in the adoption of greener energy habits, thus paving the way for a technology prototype, supporting the above.

\subsection{Current and past efforts, supporting adoption of greener habits}
A number of initiatives, some only Danish, other common European, have been deployed to reduce both the general energy-usage, and in scope of this article, with specific focus on electricity. The former organization \textit{Elsparefonden} had a primary focus of saving electricity in private households along with the public sector, but has been succeeded by \textit{Go' energi} witha more broad focus on energysavings in general \cite{ing_elspar} 

\subsection{Related work}
The hypothesis of \textit{Induced Innovation} presented by Hicks \cite{hicks}, basically suggests a relation between the demand of a potential scarce or priced resource, and the innovation in technology capable of better utilization of the resource. 
An example of this is presented, assessing energy-using consumer durables \cite{newell}. The example maps development of air-conditioning systems in terms of energy efficiency, and relates it to the previous decades increase in energy prices, and argues that the increase in energy-cost has pushed energy-saving technology.\newline

The approximately 5 \% annual increase in Danish energy prices is believed to have contributed to pushing technological innovation ahead, and a steeper increase in energy-prices might drive some innovation at a faster pace than currently.
On the other hand, what is needed in Denmark is not only technology, but also awareness and accept of the fact that fluctuations in energy-availability may occur, as a shift towards more and more renewable energy-sources happens.\newline

A Swedish research article, attempting to map what policy change instruments have impact on residential energy behavior \cite{swedish}, was published in 2006. The article analyzes different attempts since the 1970's, to regulate residential energy-habits. 
When using \textit{economic measures} such as pricing and taxing, an impact can be measured. Also using \textit{information}, like energy-labeling (similar to the EU energy-mark presented in this article) has an impact, though it seems to be somewhat delayed. Introducing \textit{physical improvements}, such as energy-meters and is observed to have a positive impact on adoption og greener habits \cite{swedish}.\newline

By gaining an overview of some of the most significant Danish energy-awareness campaigns it is believed that some of the characteristics, though not categorized similarly to the Swedish article, can be found, and give an indication of whether efforts in introduction of smartphone and IT-aided tools, for enforcing energy-awareness will be feasible.

\section{Theoretical considerations}
Attempting to enforce new technology aided habits in a population can be described by models of technology adoption. 
Traditional approaches suggest that technology adopters gain interest in a product by acquiring it and assessing its application, hereby assimilating knowledge about it \cite{gilbert}.

\subsection{Rogers innovation adoption model}
Adoption of innovations, where innovations are defined as an idea, practice or object suited for adoption, is well described by Everett Rogers diffusion research.

\subsubsection{adopter categories}
Initially Rogers models \cite{rogers_model} operates with a model describing market penetration described by an S curve, also known as the logistic function, ~\ref{fig:adoption}
The five categories are described in table ~\ref{table:rogers_char}. \newline

\begin{figure}
%\includegraphics{adoption_rates.pdf}
\includegraphics[width=0.4\textwidth]{adoption_rates.png}
\caption{\textit{Successive adoption groups each adopting technology at different rates.}}			
\end{figure}
\label{fig:adoption} % is used to refer this table in the text


%table below
\begin{table}[ht]

\caption{Adopter grouping according to Rogers model} % title of Table
\centering  % used for centering table

\begin{tabular}{|p{2cm} |p{5.5cm}|} % centered columns (4 columns)
\hline\hline                        %inserts double horizontal lines

Group & Charachteristics\\ [0.5ex] % inserts table 
%heading
\hline                  % inserts single horizontal line

Innovators &			%new line
This group is characterized by being the first individuals to adopt 
an innovation, thus this group represents only 2.5\% of the consumers \cite{diffusion2}. 
Innovators are typically members of the higher social classes, 
risk willing, has good financial lucidity, and are very social competent, 
often in close contact with other innovators. 
The innovators are seldom impacted by the fact the their risk-willingness results in an acquisition of a failed or faulty technology \cite{rogers_model} 
Innovator may even being wiling to use a technology prototype.  \\ % inserting body of the table
\hline

Early adopters & 
The early adopters are the second fastest group to adopt an innovation. What characterizes Early Adopters are a high degree of opinion leadership, compared to the other adopter groups. Innovators and early adopters are closely related in terms of education level, financial lucidity and social skills. Early adopters are a bit more judicious in their decission making regarding adoption of a new technology, compared to the innovators, thus seeking to adopt a somewhat mature technology. Their judicious approach often position early adopters in central communication positions, as opinion makers \cite{rogers_model}.
Early adopters make up about 13.5\% of the consumers \cite{diffusion2} \\
\hline

Early majority &  
This group, representing 34 \% of consumers \cite{diffusion2} are the careful consumers, avoiding the big risks, by relying on feedback from the early adopter group. 
This group seldom represents the opinion makers, but rely on feedback and recommendations from others. Their more awaiting approach to adoption results in a somewhat later time for adoption of emerging innovations\cite{rogers_model}.\\
\hline

Late majority & 
The late majority also represents app. 34\% of the consumers\cite{diffusion2}, but with a somewhat more skeptical approach. Here, only innovations that has became more common items are adopted. The late majority group is characterized, in general, by having below average social status, and only very little opinion leadership\cite{rogers_model}.\\
\hline

Laggards & 
Laggards are the remaining 16\% of customers trying to avoid changes and having to adopt new innovations. This group may not adopt a new innovation until traditional alternatives are no longer available  cs\cite{diffusion2}. Focus is very much on traditions, the group tends to have the highest age among adopters, lowest financial fluidity, and primarily only being in contact with close friends and family. Thus this group does not represent any opinion leadership\cite{rogers_model}\newline \\ 
\hline

\end{tabular}
\label{table:rogers_char} % is used to refer this table in the text
\end{table}


\subsubsection{Rogers five step innovation diffusion process}
Adopting an innovation requires a number of decisions. The decisions happens sequentially over time, and are by Rogers \cite{rogers_model} described as a five stage process.
\begin{figure}
\includegraphics[width=0.45\textwidth]{iv_process.png}
\caption{\textit{The innovation decision process}}			
\label{fig:iv_process} % is used to refer this table in the text
\end{figure}


Briefly the five steps are :

\textit{Knowledge:} When the individual is exposed to the innovation,  without prior knowledge .

\textit{Persuasion:} The individual has gained interest, and seeks further information on the innovation.

\textit{Decision:} Weighing the advantages / disadvantages of the innovation, thus rejecting or adopting the innovation initially. 

\textit{Implementation:} The innovation is employed, and its usefulness assesed, possible search for further information may occur.

\textit{Confirmation:} Final assessment of the innovations usefulness, thus determining futore use and full utilization.   

Common for all five stages are that transition from one stage to another requires information to flow through \textit{Communication channels}. Likewise a \textit{Social systems}, must be present to frame the model.\newline


\subsubsection{a modified model of the innovation diffusion process}
During a study, that has tried to identify the factors influencing the massive emerge of photovoltaics, Abram Kaplan, an American researcher has been attempting to asses the factors behind the rapid emerge, using Rogers model for \textit{diffusion of innovation}.
Kaplan strongly argues and proves that Rogers five factors, introduced in the previous section, in not operational enough as a model, and is lacking a few factors. Kaplan presents and proofs a model, inspired from Rogers, but it has following key differences, in the first two of the five steps, before the third step \textit{Decision} \cite{passive_to_active}: \newline

\textit{Decisions does not happen sequentially}, but instead, the most vital factor resulting in adoption is \textit{interest}. \textit{Interest} is stimulated by a number of variables, inspired by Rogers theory. \newline

\textit{Development of new variables}: \newline

\textit{Context}, similar to Rogers, and describes exogenous variables ("outside factors, that are unchangable or independent"), under three subcategories, \textit{Enviromental, Organizational} and \textit{Personal}. These outline the scope and affect the social system, that must be present to use the model.\newline

\textit{Motivation}, may be a good fertilizer of interest, but is not sufficient alone. Motivation can be both ideological, in the case of photovoltaics, sustainability and independence may be the motivation, but also economic factors, such as payback time can be a motivation. \newline

\textit{Knowledge} is still central, but does not cover as broad as Rogers model, in this context, knowledge refers to a factual, objective base of information, more specific the term \textit{Technical Knowledge} is used.\newline

\textit{Experience}, can be both good and bad, but is thought to be rather vital, experience can contribute to knowledge and behavioral intentions and actual adoptions.\newline

\textit{Familiarity}, a cognitive state resulting from experience. Familiarity suggest a level of comfort with the object, that cannot result from only \textit{Knowledge}. Often \textit{brand loyality} can positively affect adoption, and this is closely associated with familarity.\newline

\textit{Interest}, is the ultimate dependent variable, leading to the adoption decision, interest is similar to Rogers \textit{Persuation}. Interest describes the likelihood of adoption. \newline


Kaplan has operationalized his model by collecting survey results from approximately 400 US adopters and non-adopters of photovoltaics. Hereby assessing the factors affecting the group of interest, has identified, as playing a role regarding their choice either to adopt the technology or not. The model has been deployed, using numbers and weights to get comparable results, thus validating or rejecting hypothesises.

Photovoltaics has gained massive interest in Denmark and Germany, in particular, so a model assesing the rapid growth, within this field is believed also to be suited, when looking into changing energy habits into something a bit more green, aided by IT-innovations.

\begin{figure}
\includegraphics[width=0.5\textwidth]{kaplan_model.png}
\caption{\textit{Kaplans modified innovation decision model}}			
\label{fig:kaplan_model} % is used to refer this table in the text
\end{figure}


\subsection{hypothesis}
\textit{H1:"There are potential uncovered needs, in visualizing energy consumption in real time, to enforce greener habits, among the groups, Innovators, Early Adopters and Early Majority, according to rogers model for innovation adoption."}\newline

\textit{H2:"By presenting a clear distinction between leisure and necessary usage, the consumer groups, capable of adopting this technology, will be able to develop a greener mindset"}

\section{model}
By using Rogers models for technology diffusion and adoption, a framework will be established to asses the impact of some of the energy saving campaigns the in the recent years.
The perspective applied is from an approach where the interest is to focus on IT aided solutions used to stimulate new greener habits among consumers.

Since the emerge of new IT aided technologies for communicating,  browsing the web, collecting data and communicating via social medias has experienced a massive growth the last decade, it will be a valid assumption that these technologies will play a central role in most Danish energy consumers everyday life.

This field, in particular the emerge of smartphones, combined with collective sensing and data-sharing presents a massive potential in terms of ability to collect, share and analyse data. Much of this relies on the continuous development and emerge of new technologies, thus models for adoption of new technology, in particular \textit{Everet Rogers } theory behind \textit{Diffusion of innovations will be used} to asses which adopter groups have been impacted by current campaigns.

Abram Kaplans model will be used as a stencil, to pinpoint which factors may have been actuated by the campaings of interest, in this article. A survey among energyusers will be outside the scope of this article, but would be an interesting follow-up, to pinpoint more accurately, the effect of the campaigns.  

Existing evaluation of present conducted campaigns and where possible evaluation of current activities, both with primary focus on electricity usage, will be analyzed.


\section{current and past initiatives}
Below four of the most significant, and regional initiatives, in Denmark across the last couple of years are outlined

\subsection{Energymarking of electric products}
\begin{figure}
\includegraphics[width=0.15\textwidth]{energy_mark.png}
\caption{\textit{EU's energymark}}			
\label{fig:emark} % is used to refer this table in the text
\end{figure}

The \textit{EU energy-mark} is mandatory for all producers and retailers of following household appliances: 
Fridge and freezers, Dishwashers, Washing-machines, Tumble-driers, Electric bulbs, Ovens, 
Air-conditioning and climate devices and Televisions.
The mark originally rates products from \textbf{A} to \textbf{G}, where \textbf{A} indicates the best energy efficiency. 
Additions allows \textbf{A+}, \textbf{A++} and \textbf{A+++} marking, if technology allows this. \cite{energymark}

The rating is relative, to the current technological availability, for instance regarding \textit{light-bulbs}, 
here the bulbs are compared to traditional incandescent lightbulbs, today an \textbf{A}-rated buld shall produce 4 times more \textit{Lumen} per \textit{Watt} than a similar incandescent type, emitting the same amount of light. For light  bulbs the energy mark also specifies the expected lifetime in \textit{hours}.

When buying new household appliances the \textit{energy-mark} is a good factual indication, that can be used to judge the
best energy efficiency, and thus biasing the customer to buy energy efficient products. 

A potential drawback of the energy mark, is the risk of a non-persistent "green-impact", since the \textit{energy-mark} 
primarily is to be used during purchase. History shows that technology and also energy efficiency improves over time, so the once purchased energymarked \textbf{A} fridge, might not be as efficient any more, when comparing to the cutting-edge, currently available. Thus the impact of the energy mark may suffer from the drawback, that it may not continuous affect the users behavior, once the purchase has been made.  

Preceding the \textit{energy-mark} is the \textit{Electricity-saving mark}, which essentially shows the efficiency of an electric household utility. The marking using categories \textbf{A} to \textbf{G} is similar for both marking systems. 
The \textit{Electricity-saving mark} had in 2009 achieved a level of customer awareness of 79 \%. The transition from \textit{Electricity-saving mark} to \textit{Energy-saving mark} has resulted in a little confusion among customers, thus the transparency, and similarity of both marks has made the transition relatively smooth. 
The \textit{Energy-saving mark} is by customers perceived as a trustworthy tool, aiding purchase of energy efficient products \cite{annual_ens_10}.

The legislation regarding \textit{energy-marking of energy-related products} states that savings are to be achieved by informing the end user about consumption during usage, and other vital facts. \cite{energylaw1} Other vital facts could be emitted \textit{Lumens} and expected liftime in \textit{Hours} for a lightbulb.\newline 


\textbf{\textit{Keywords - efficiency, lifetime, standardization, harmonization, trustworthy.}}

\subsection{Go' Energi}
\begin{figure}
\includegraphics[width=0.15\textwidth]{enok.png}
\caption{\textit{Go' energi's native Eskimo}}			
\label{fig:enok} % is used to refer this table in the text
\end{figure}

Go' Energi is an independent state initiated campaign targeted at Danish House-owners, with the aim of informing about potential improvements within: indoor climate, comfort, energy economics and climate awareness \cite{goenergi}.

The campaign has selected an Eskimo, a member of the Greenlandic population as a symbol. The Eskimo symbolizes harmony with nature, passion and conscience for the nature and environment, and not least, the Eskimos has among any population, great skills and knowledge about keeping warm in a cold climate. Greenland 
is an autonomous country within the kingdom of Denmark, hereby the Danish population have some kind of relation to an Eskimo.  
\textit{"Keeping warm, in a conscious way"} is more or less the mindset the campaign aims to spread among the Danish house-owners. 

The campaign was established in 2010, and is led by \textit{Center For Energy-savings}, an organization founded by the Ministry of Climate- and Energy. Funding of the campaign and the organization is a tax of 0,006 DKK/KWh, paid by customers in the households and public sector. \textit{Center For Energy-savings} succeeds the former orginasation \textit{Elsparefonden}, who had a dedicated focus on saving electricity \cite{annual_ens_10}. 

Among the instruments used to promote energy improvements are following:

\textit{Selected product tests:} Selecting products of interest, and verifying whether or not they match their energy-mark rating. Few products, for example air-heat-pumps, have been much to optimistic rated in terms of energy-efficiency. By conducting sample tests, producers and retailers will gain increased awareness and attention to the specifications of the products they sell. Also this will have the benefit of sharpening discipline, to retain an energy-friendly and trustworthy brand in the relations between producers/retailers and customers.

\textit{Promoting the 20\% most energy efficient products:} Through assessing new product data and conducting tests ,the 20\% best performing products, in terms of energy efficiency, will be awarded the energy mark from \textit{Go energi}, indicating they provide a good energy performance. The \textit{Go energi} mark is hoped to gain the same trustworthy perception among customers as the \textit{EU energy-mark}. 

\textit{Educational training targeted at resellers and craftsmen:} Campaigns are being conducted to educate and raise the knowledge of reseller-staff, thus 
improving their skills, concerning their ability to guide and advise consumers, regarding purchasing energy efficient products.
Also training targeted at craftsmen, such as carpenters and electricians, are conducted. This to leverage the ability of competent guidance, and presenting energy saving, well proved mature solutions, when renovating older buildings, or crafting new buildings \cite{annual_ens_11}.  

Besides from the above, \textit{Go energi} provides evaluations, tools, and training for  public sectors, institutions etc. beyond the scope of this \newline

\textbf{\textit{Keywords - inspiration, optimization, environmental conscience, targeted information, testing, recommendations.}}

\subsection{Behavioural campaign - "Chasing the hidden energy"}
Actually \textit{Chasing the hidden energy} campaign is targeted at public institutions, where the aim of this article is focused at private households. 

\textit{Chasing the hidden energy} asses the fact that approximately 50\% of energy-usage within the public sector is consumed by two equally energy-consuming groups of products, one being electric lights and the other being IT equipment and other electronic equipment\cite{bk_2004}.
It is argued that a massive saving potential, up to 20\%, can be realized, by changing behavior, in terms of acting responsible, turning of the lights, remembering to shut down PC's, coffee-machines, printers etc., at the end of the day. 
Further savings between 10\% and 25\% can be achieved by smaller investments in energy saving equipment, such as \textit{Auto-power-off electric sockets, led bulbs, etc.}\cite{hidden_e}. 

Changing behavior, rewarding achieved goals, setting the goals and gaining commitment all requires leadership, hence an emerging trend is \textit{Energy-leadership}. 

To perform successful energy-leadership following prerequisites are essential.

\textit{A clear visualization of the current energy use :} In order to achieve goals, and being able to track the results of ongoing efforts, it is vital to have as clear intuitive visual presentation of the energy usage.
 
\textit{A clear strategy :} Formulate a clear strategy that all \textit{members of the community} can support, this can be employees, residents in a local housing association, or just your family members. The strategy should formulate few ambitious, but realistic achievable goals. 

\textit{Commitment :} Strategy and visualization can only become operational and useful if the participants are committed, motivated, and can associate with the goals. An ambassador should be appointed, in an institution, preferably a non-formal employee, a person that has respect among colleagues, and also possess social skills. Relating to \textit{Rogers segmentation of technology adoption groups}, the ambassador should be in the \textit{Early Majority} segment, thus enjoying respect from the somewhat more conservative \textit{Late Majority} and \textit{Laggards}. If the ambassador is to fuzzy and has the characteristics of an \textit{Innovator}, commitment from the late segments may be hard to achieve. 

Commitment can be realized by rewarding good behavior, this can be small things like \textit{"the boss buys cake when all printers and light have been turned off for two weekends"} or perhaps state that 5\% of the annual saving are used to fund an event like a summer-party 

The effect of an energy-campaign in a company is believed to have a mutual positive effect on the participants energy-habits in their homes\cite{hidden_e}. Energy-leadership can easily be scaled down and applied in private households. \newline

\textbf{\textit{Keywords - leadership, reward, visualize, smartness.}}


\subsection{Se' elforbrug}
Se-elforbrug is a web-tool, mandatory for public institutions and optional for factories and other larger electricity consumers. The tool prerequisites that the installation is equipped with remote-read energy meters \cite{see}. 
Once participating, the tools monitors consumption on an hourly basis, and presents graphs and trends in the consumption. Consumption can be grouped based on affiliation, and compared to other similar users, e.g. \textit{schools} can comped against each other, in realizing the lowest energy usage.
Companies, municipalities or other institutions that commits to a reduction in energy-usage or Co2 emission can visualize their real-time data to the public, thus committing to their obligations.

Similar initiatives are emerging availible to the private customer, with a \textit{smart-meter} equipped installation. For instance \textit{Energi-midt} have developed a smartphone app. capable of displaying historical electricity usage, with an hourly granularity \cite{em_app}. \newline

\textbf{\textit{Keywords - data-collection, optimization, compare, commitment, real-time data.}}

\section{scope}
This article asseses some of the current attempts to reduce energyconsumption among danish energyconsumers.

The article will be written also in the scope of the Ecosense project, focus will be on the potential opportunity to visualize energy consumption in typical households. A clear distinguishing between mandatory and leisure usage, is believed to reinforce greener habits, when presented clear and intuitive.

The article is written as part of the publications within the Ecosense project. The Ecosense project is a collaboration between several Danish universities, companies and foreign research institutions. 

The common emphasis in Ecosense is on using smart-phones and collective sensing, to visualize the energy usage, and impact on the environment, aiming at reinforcing new and greener habits. 

The Ecosense project is funded by the Danish Strategic Research Council.

\section{Rating method}
A simple framework for comparing the initiatives is presented below:
The following variables are calculated and plotted in an X-Y chart: \newline

\textbf{Popularity}: is based on the increasing adoption decision factors achieved, in table ~\ref{table:asses_init}, the factors in figure ~\ref{fig:kaplan_model} are assigned values increasing, as the factor gets closer to stimulating interest.
Thus \textit{Motivation} and \textit{Context} has the value \textbf{10}, \textit{Experience} and \textit{Knowledge} the value \textbf{20} and \textit{Familarity} the value \textbf{40}. A factor is multiplied with its rating, regarding stimulation of the adoption factors, being \textit{low} weighing \textbf{1}, \textit{medium} weighing \textbf{2}, and \textit{high} being \textbf{3}.
Thus popularity will be made up of a sum of all five factors, multiplied with their rating.\newline

\textbf{Maturity}: indicates the number of segments achieved, according to Rogers segmentation of adopter groups. 
To get a rating, the segment or segments reached, has the center-weight of the rightmost group, in the logistic curve figure ~\ref{fig:iv_process}, e.g. \textit{late majority gets the score 75}, this will provide a good measure of the targeted audience.\newline

\textbf{Current Phase}: describes the current phase, along with the authors perception on, where in its life-cycle the campaign currently resides. Based on its Current Phase, being \textit{Growing, Stable} or \textit{Declining}, a score is assigned, being the value \textbf{1} for \textit{Declining}, \textbf{3} for \textit{Stable} and \textbf{10} for \textit{Increasing}. The exponential increase in value, 1, 3 and 10 accordingly, represents the fact that market-penetration of new technologies follows an Exponential patterns, known as the \textit{S-curve}\cite{rogers_model}, and hereby the value is derived from the slope of the curve.  \newline

\textbf{Potential}: attempts to look into the crystal ball, calculating a score that aims at predicting the future for the selected campaign. The score is calculated based on a multiplication and weighing of the three previous factors. \newline 


%table below
\begin{table}[t2]

\caption{Assigning scores for different initiatives} % title of Table
\centering  % used for centering table

\begin{tabular}{|p{2.5cm} |p{0.9cm} |p{0.9cm} |p{0,9cm}| p{0,9cm} |} % centered columns (4 columns)
\hline\hline                        %inserts double horizontal lines

Initiative & Maturity & Popularity & Current Phase & Potential\\ [0.5ex] % inserts table 
%heading
\hline                  % inserts single horizontal line

EU energy mark & 
98  & 
230 &
Stable &
62 \\
\hline

Go' energi & 
75 & 
190 &
Declining &
14 \\
\hline

Chasing the hidden energy & 
40 & 
270 &
Growing &
108 \\
\hline

Se-elforbrug & 
15  & 
160 &
Growing &
44 \\
\hline
\end{tabular}
\label{table:score} % is used to refer this table in the text
\end{table}

Table ~\ref{table:score} summarizes the perception and tries to present predictions for these campaigns. The results are mapped in a coordinate system, figure  ~\ref{fig:xy},  helping to visualize the perception of the campaigns.

\begin{figure}
\includegraphics[width=0.5\textwidth]{xy.png}
\caption{\textit{Visualization of the evaluated campaigns}}			
\label{fig:xy} % is used to refer this table in the text
\end{figure}


It can be seen that the mature initiative \textit{EU energy-mark} still remains a solid benchmark, and since perceived trustworthy by users, it seems obvious to maintain this marking format.
\textit{chasing the hidden energy}, is expected to be a rising star, much since no costs or purchase are associated with "getting started", it is just a matter of attending, and starting to adopt new habits.
\textit{Se-elforbrug} is still a bit minded at "data-mining-nerds", often an energy-consultant, or responsible for energy improvement, in public institutions, with technical capabilities.
\textit{The Go' energy} campaign has had a big area of interest to asses, and has presented many different initiatives and test, aiming at providing an overview. The decision to close \textit{Go' energy} seems a bit hasty, since the presented initiatives has probably not had enough time to gain sufficient momentum, to impact energy-users, enough.     



\section{Discussion}
In order to asses the initiatives presented in the previous section, the key characteristics of each campaign, will be assessed and trying to match specific keywords from Rogers and Kaplans models presented.
The result of the assesment and rating is seen in table ~\ref{table:asses_init}.

\subsection{Scoring method, adoption factors}
Table ~\ref{table:asses_init} assess the four initiatives, in terms of pointing out whether one or more groups of potential adopters, according to Rogers segmentation, have been targeted during the campaign. 
Furthermore, the five stimulation factors, identified in Kaplans modified model, are used as measuring points, thus aiming at identifying whether the campaigns message has a potential for stimulating adoption. 

\subsubsection{charachteristics}
The second colums in table ~\ref{table:asses_init}, is a summary, of the keywords identified previously, describing the characteristics of each of the four campaigns

\subsubsection{grouping into adopter segments}
In table ~\ref{table:asses_init}, in the third column, the different initiatives are assessed, and characterized among each other. Characterization is done in relation to the model attempting to identify one or more segments the campaign has successfully targeted. 

\subsubsection{stimulating adoption}
The fourth column in table ~\ref{table:asses_init} attempt to rate the campaigns aim, at each of Kaplans factors, stimulating interest, preceding an adoption decision. The score given can be \textit{low, medium} and \textit{high}

%table below
\begin{table*}[t2]

\caption{Assessing different initiatives} % title of Table
\centering  % used for centering table

\begin{tabular}{|p{2,5cm} |p{3.5cm} |p{2.5cm} |p{4.5cm} |} % centered columns (4 columns)
\hline\hline                        %inserts double horizontal lines

Initiative & Charachteristics & adopter group, according to Rogers & Stimulation of adoption factors according to Kaplan\\ [0.5ex] % inserts table 
%heading
\hline                  % inserts single horizontal line

EU energy mark & 
Mandatory, trustworthy, harmonization, Easy understandable, Rational  & 
All &
Motivation  [high] \newline 
Context	    [medium] \newline 
Knowledge	[low] \newline 
Experience  [medium] \newline 
Familiarity [high] \\
\hline

Go' energi & 
Achieve savings by adopting new habits, Informative, Buyer guidance, optimization, & 
Early majority\newline  Late Majority &
Motivation  [medium] \newline 
Context	    [high] \newline 
Knowledge	[medium] \newline 
Experience  [low] \newline 
Familiarity [medium] \\
\hline

Chasing the hidden energy & 
Educational, Reveal "hidden" consumption, Reward initiative, Promote energy-leadership & Early Adopters\newline Early majority &
Motivation  [medium] \newline 
Context	    [high] \newline 
Knowledge	[medium] \newline 
Experience  [high] \newline 
Familiarity [high] \\
\hline

Se-elforbrug & 
Followup, guidance, compare and rate, visualize savings, optimization  & 
Early Adopters &
Motivation  [medium] \newline 
Context	    [medium] \newline 
Knowledge	[high] \newline 
Experience  [low] \newline 
Familiarity [low] \\
\hline
\end{tabular}
\label{table:asses_init} % is used to refer this table in the text
\end{table*}

\subsection{Arguing table ~\ref{table:asses_init}}
This article has primarily focused at four recent initiatives, implemented in Denmark, where two of the initiatives is actually targeted at sectors outside the area of interest, the households. The reason for this, is that the campaign \textit{Chasing the hidden energy} tries to change the mindset, at peoples job, but aiming at this mindset-change to be deployed in their private households as well. The tool \textit{Se-Elforbrug} is interesting because it displays some of the possibilities leveraged, to users, when deploying remote and collective sensing, such as data-sharing, collective measuring, enforcing companies and institutions to comped against each other, in lowering consumption.\newline

In the following sections, the rationale behind the scores in table  ~\ref{table:asses_init}
will be presented and discussed.

\subsubsection{The EU-energy-mark}
The EU-energy mark has been not been around for more than a couple of years, but since it is closely related to its predecessor, the \textit{electricity-saving mark}, it has the advantage being familiar, to most consumers. Its rational and non-commercial information, communicating through easy understandable symbols, poses an advantage to all adopter-groups, thus not requiring much background knowledge, to retrieve its intentional information.

\subsubsection{The Go-energy campaign}
Go-energy has succeeded all initiatives from the foprmer \textit{Elsparefonden}, thus broadened its focus, from bing targeted at electricity savings, to cover energy savings in general. Go-energy fits well in the Danish context, where the government has a very strong focus on achieving climate goals, a milestone of a 40\% CO2 reduction in the year 2020 compared to 1990. The ambitious goals presented, still posses challenges in terms on educating the public to act more responsibly, also a number of taxes have been introduced the last decade to increase the economic motivation, realized by saving energy. 
Plans to shut down \textit{Go-energi} during 2012 \cite{luk_ge} persists, after the recently adopted \textit{Energy-agreement} \cite{energipolitik_2020}. After this, the activities led by \textit{Go-energi} and previous \textit{Elsparefonden} are to be continued by the Danish Utility-companies.
Leaving this effort to the Utility-companies could potentially have the drawback of resulting in a fragmented effort, without the necessary transparency and uniformness presented by \textit{Go-energi} sofar. Also the confidence of an independent, external funded organization, as perceived by user, will be lost when the utility-companies needs to continue this effort. 

\subsubsection{Chasing the hidden energy}
Campaigns aiming at behavioral change always requires motivated users to become successful. Fortunately current and past efforts, green-taxes, and the Danish cleantech and environmental focus has created both awareness and knowledge among the Danish population, thus supporting a solid context for behavioral change. Campaigns deployed in common, at workplaces, if successfully adopted and perceived would be assumed to stimulate both \textit{Experience} and \textit{Familarity}, thus this seems like a good way of paving the way of adopting supporting technology and habits. 

\subsubsection{Se-elforbrug}
Most likely a large variety of tools, aiding users in visualizing their energy usage will emerge. Common for most of these including \textit{Se-elforbrug}, is a required interest in visualizing consumption and available technical knowledge. Probably these are some of the factors affecting the choice of target-audience, the industry-consumers and public institutions.
The author of this article is a strong believer that the massive increase in emerge of smart-phones, and mobile data-coverage with flat-rate data-packages combined with ever increasing smartness and connectivity being deployed in more and more household and everyday products, will continue to leverage the learning-barrier and flatten the steepens of the technological learning-curve. Also the pervasive nature of these technologies will positively affect the entire populations \textit{experience} and \textit{familiarity}, resulting in a more "technology-adoption-ready" audience.


\section{Conclusion and further work}
Many different tools, campaigns, public and private funded, and volunteer initiatives are being deployed, to pay attention to the fact that we have to use energy in a more responsible manner. Not only in Denmark, which is the focus of this article, but globally,  among some of the "big-spenders" like the U.S., are energy and climate awareness becoming part of the agenda. \newline

In the above chapters, four different Danish initiatives have been presented, all with the common focus of informing energy-users. An exhaustive presentation of all available energy- informing \& reducing initiatives is beyond the scope of this article. The four campaigns presents their corresponding messages using different instruments. The presented initiatives aim to reach their target audience through different communication channels, such as marking in stores, advertising campaigns, education and by providing tools and instruments.

Education and availability of IT-tools, are currently often only presented to the public sectors, and the industry, since these are the overall \textit{big-consumers}. Different options are though available for the private household-consumer 

Looking at hypothesis 1, that states :
\textit{"There are potential uncovered needs, in visualizing energy consumption in real time, to enforce greener habits, among the groups, Innovators, Early Adopters and Early Majority, according to rogers model for innovation adoption."}
reveals that adopting practices from the \textit{Se-energi} tools, and leveraging it to the household-customer, with adequate presentation and guidance would probably stimulate new energy-friendly habits. This is the experience, and hope, that educational campaigns like \textit{Chasing the hidden energy} succeeds with.
If success is achieved in enforcing new habits, the marked for IT-aided tools, like smartphone-apps, that helps the user, will definitively be mature of adopting initiatives presented.  

It is believed that campaigns aimed at saving energy, through good habits, will have effect at home, even though presented and enforced only at work. Combining the latter with the popularity and perceived trustworthiness of the \textit{EU energy-mark} leads to the perception that the second hypothesis: 
\textit{"By presenting a clear distinction between leisure and necessary usage, the consumer groups, capable of adopting this technology, will be able to develop a greener mindset"} will invite the users to think about the actual composition of their energy consumption, thus addressing issues like \textit{standby-consumption}, \textit{"leaving the lights on"}, \textit{teaching and rewarding the kids to turn of their gaming consoles when done playing"}. If \textit{true energy-awareness} can be achieved among energy users and customers, it will also have a positive effect on the perception and willingness to participate in deploying the future inevitable renewable-compatible infrastructure.   \newline

The data-base, on which the model has been build and executed, consists only of findings, numbers and quotes, retrieved from relevant sources. Also the model that calculates and ranks \textit{maturity, popularity} and \textit{potential} is rather simple. A more comprehensive framework, together with a formal questionnaire, sent out in a larger scale, would probably present a more nuanced perception of the current ongoing initiatives.

The theory behind \textit{adoption of innovations} by Rogers seems well suited for segmenting energy-users into manageable groups with distinct characteristics, thus allowing different initiatives to be targeted specifically at each adopter-segment.

Kaplan's modified model of \textit{Innovation Decision} provides some excellent measuring points, for stimulating interest in an innovation. When designing and deploying new innovations, such as a Smartphone-App or a prototype, a measure or at least a checklist addressing Kaplans variables will be very usefull.




\newpage
% trigger a \newpage just before the given reference
% number - used to balance the columns on the last page
% adjust value as needed - may need to be readjusted if
% the document is modified later
%\IEEEtriggeratref{8}
% The "triggered" command can be changed if desired:
%\IEEEtriggercmd{\enlargethispage{-5in}}

% references section

% can use a bibliography generated by BibTeX as a .bbl file
% BibTeX documentation can be easily obtained at:
% http://www.ctan.org/tex-archive/biblio/bibtex/contrib/doc/
% The IEEEtran BibTeX style support page is at:
% http://www.michaelshell.org/tex/ieeetran/bibtex/
%\bibliographystyle{IEEEtran}
% argument is your BibTeX string definitions and bibliography database(s)
%\bibliography{IEEEabrv,../bib/paper}
%
% <OR> manually copy in the resultant .bbl file
% set second argument of \begin to the number of references
% (used to reserve space for the reference number labels box)

\begin{thebibliography}{1}

\bibitem{passive_to_active}
A.W. Kaplan, \emph{From passive to active about solar electricity: innovation decision
process and photovoltaic interest generation}, \relax Environmental Studies Program, Denison University, Granville, OH 43023, USA, July 1998.

\bibitem{udv_elpriser}
M. Togeby, A. E. Larsen, \emph{Udviklingen af elpriserne}, \relax Ea Energianalyse a/s Copenhagen Denmark, Feb.2011.

\bibitem{energistat}
Energistyrelsen, \emph{Energistatistik 2010}, \relax Energistyrelsen, Copenhagen Denmark, Sept.2011.

\bibitem{energipolitik_2020}
\emph{Aftale mellem regeringen (Socialdemokraterne, Det Radikale Venstre, Socialistisk Folkeparti) og Venstre, Dansk Folkeparti, Enhedslisten og Det Konservative Folkeparti
om den danske energipolitik 2012-2020} \relax Folketinget March 2012

\bibitem{rogers_model}
Rogers, E.M., \emph{Diffusion of Innovations}, \relax 4th ed. Free Press, New York. 1995

\bibitem{diffusion2} 
acb \emph{product diffusion curve},\relax www.quickmba.com/markting/product/diffusion

\bibitem{gilbert} 
Gilbert, M., Cordey-Hayes, M., 1996. \emph{Understanding the process of
knowledge transfer to achieve successful technological innovation}, \relax Technovation 1996

\bibitem{energylaw1} 
\emph{Lov om energimærkning af energirelaterede produkter1}, \relax Klima- og Energiministeriet, Copenhagen May. 2011 

\bibitem{energymark} 
\emph{European Energy Mark}, \relax Klima- og Energiministeriet, Copenhagen May. 2011 

\bibitem{goenergi} 
\emph{Go' energi campaign}, \relax http://www.goenergi.dk/presse/presseservice/kampagner/enok-energibesparelser-i-bygninger-2011/om-enok-kampagnen 

\bibitem{seElforbrug}
\emph{Se Elforbrug}, \relax www.goenergi.dk/publikationer/vejledninger/se-elforbrug, Copenhagen Dec. 2012

\bibitem{annual_ens_10} 
Center for Energibesparelser \emph{Årsberetning 2010}, \relax http://www.goenergi.dk/publikationer/handlingsplaner/aarsberetning-2010

\bibitem{annual_ens_11} 
Center for Energibesparelser \emph{Årsberetning 2011}, \relax http://www.goenergi.dk/publikationer/handlingsplaner/aarsberetning-2011

\bibitem{ing_elspar}
Avisen Ingeniøren \emph{Elsparefonden blev nedlagt med et pennestrøg}, \relax 
http://ing.dk/artikel/101572-kritikere-elsparefonden-blev-nedlagt-med-et-pennestroeg, Copenhagen Aug. 2009

\bibitem{bk_2004}
Energistyrelsen \emph{Potentialevurdering Energibesparelser i husholdninger, erhverv og offentlig sektor}, \relax http://www.klimakompasset.dk/files/pdf


\bibitem{hidden_e}
Ditte Vesterager Christensen \emph{Energirigtig adfærd vha. kampagne}, \relax Go' energi Oct. 2011.

\bibitem{see}
\emph{Se Elforbrug}, \relax http://www.goenergi.dk/offentlig/vaerktoejer-og-beregnere/se-elforbrug

\bibitem{em_app}
\emph{Mit Energi}, \relax 
http://www.energimidt.dk/Privat/El-vand-og-varme/Forbrugsvisning/Sider/Foelg-dit-elforbrug-via-din-smartphone.aspx 

\bibitem{luk_ge}
\emph{Go energi lukkes efter energiforlig}, \relax
http://jp.dk/indland/indland\_politik/article2731282.ece, March 2012

\bibitem{newell}
R.G. Newell et.al. \emph{THE INDUCED INNOVATION HYPOTHESIS AND ENERGY-SAVING TECHNOLOGICAL CHANGE}, \relax The Quarterly Journal of Economics, August 1999

\bibitem{hicks}
Hicks, John R., \emph{The Theory of Wages}, \relax London: Macmillan, 1932

\bibitem{swedish}
Linden, Anna L. et.al., \emph{Efficient and inefficient aspects of residential energy behaviour: What are the policy instruments for change?}, \relax Energy Policy
Volume 34, Issue 14, September 2006


\end{thebibliography}




% that's all folks
\end{document}


