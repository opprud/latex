%
\documentclass[conference]{IEEEtran}
\usepackage{graphicx}
\usepackage{placeins}
%\usepackage{caption}
\graphicspath{{./figures/}}



% *** PDF, URL AND HYPERLINK PACKAGES ***
%
%\usepackage{url}
% url.sty was written by Donald Arseneau. It provides better support for
% handling and breaking URLs. url.sty is already installed on most LaTeX
% systems. The latest version can be obtained at:
% http://www.ctan.org/tex-archive/macros/latex/contrib/misc/
% Read the url.sty source comments for usage information. Basically,
% \url{my_url_here}.







% correct bad hyphenation here
\hyphenation{op-tical net-works semi-conduc-tor}


\begin{document}
%
% paper title
% can use linebreaks \\ within to get better formatting as desired
\title{Implementing energy usage visualization in ?? segment in the danish Energy System}


% author names and affiliations
% use a multiple column layout for up to three different
% affiliations
\author{\IEEEauthorblockN{Morten Opprud Jakobsen}
\IEEEauthorblockA{AUhe\\
Aarhus University Herning\\
Business and social sciences\\
Birk Centerpark 15 - 7400 Herning\\
Email: morten@hih.au.dk}}





% use for special paper notices
%\IEEEspecialpapernotice{(Invited Paper)}




% make the title area
\maketitle


\begin{abstract}
%\boldmath
A 90\% reduction in the emission of climate gases, is the ambitious Danish goal, presented to achieve the maximum temperature rise of 2$\,^{\circ}$C  in the global average temperature. This goal cannot solely be achieved by increasing energy taxes and optimizing energy usage, a vital mean to reduction is energy awareness. 

This article asseses some of the current attempts to reduce energyconsumption among danish energyconsumers.

In the scope of ecosense, focus will be on the opportunity to visualize energy consumption in typical households. A clear distinguishing between mandatory and leisure usage, is believed to reinforce greener habits, when presented clear and intuitive.

\textit{Keywords - awareness, visualization, smart metering, greener habits}
\end{abstract}

\section{scope}
This article is written as part of the publications within the Ecosense project. The Ecosense project is a collaboration between several Danish universities, companies and foreign research institutions. 

The common emphasis is on visualizing the individuals energy habits, and impact on the environment, aiming at reinforcing new and greener habits. 

The Ecosense project is funded by the Danish Strategic Research Council with 18 millions.
 
\hfill Herning, Summer 2012

\section{Introduction}
% no \IEEEPARstart
Electricity has become a necessity in all developed countries, not only in Denmark on which this article focuses. Yet with fluctuating production prices, due to a growing share being produced by renewable sources, and a long term trend of an annual increase of approximately 5 \% \cite{udv_elpriser}, still private households acounts for 20\% of the total electricity consumption in Denmark \cite{energistat}.
The danish governments ambitious goal Denmark being a pioneering nation, that must be self-sustainable, having the entire energy comsumption covered by renewable energysources by 2050 \cite{energipolitik_2020} 

The long term increase in disposable income increases to leverage the availability of electrical appliances in Danish households %\cite{keylist}. 

The increase in energy consuming appliances can be split into both leisure-usage products, such ad flat screen TV's gmaing consoles, laptops etc. labeled \textit{wants}, and on the other hand white goods and other home appliances, aiding the busy family in their daily business, labeled \textit{needs}.

.....Argue that visualization og the actual energy usage, and the distribution between \textit{needs} and \textit{wants} may actuallty stimulate awareness and interest in the actual energy usage.

...will try to present a framework for stimulation adoption of greener energy habits.1




\subsection{Current and past efforts, supporting adoption of greener habits}

\subsection{development of energyconsuming devices in households}

\section{Theoretical considerations}
Attempting to enforce new technology aided habits in a population can be described by models of technology adoption. 
Traditional approaches suggest that technology adopters gain interest in a product by acquiring it and assessing its application, hereby assimilating knowledge about it \cite{gilbert}.

\subsection{Rogers innovation adoption model}
Adoption of innovations, where innovations are defined as an idea, practice or object suited for adoption, is well described by Everett Rogers diffusion research.
Initially Rogers models \cite{rogers} operates with a model describing market penetration described bu an S curve, also known as the logistic function.


\begin{figure}
%\scalebox{8cm}{!}{\includegraphics{adoption_rates.png}}
% %\includegraphics[bb=0 0 2 18]{adoption_rates_2.png}
\caption{\textit{Successive adoption groups each adopting technology at different rates.}}			
\end{figure}


%table below
\begin{table}[ht]

\caption{Adopter grouping according to Rogers model} % title of Table

\centering  % used for centering table

\begin{tabular}{|p{2cm} |p{5.5cm}|} % centered columns (4 columns)

\hline\hline                        %inserts double horizontal lines

Group & Charachteristics\\ [0.5ex] % inserts table 

%heading

\hline                  % inserts single horizontal line

Innovators &			%new line
This group is characterized by being the first individuals to adopt 
an innovation, thus this group represents only 2.5\% of the consumers \cite{diffusion2}. 
Innovators are typically members of the higher social classes, 
risk willing, has good financial lucidity, and are very social competent, 
often in close contact with other innovators. 
The innovators are seldom impacted by the fact the their risk-willingness results in an acquisition of a failed or faulty technology \cite{rogers_model} 
Innovator may even being wiling to use a technology prototype.  \\ % inserting body of the table
\hline

Early adopters & 
The early adopters are the second fastest group to adopt an innovation. What characterizes Early Adopters are a high degree of opinion leadership, compared to the other adopter groups. Innovators and early adopters are closely related in terms of education level, financial lucidity and social skills. Early adopters are a bit more judicious in their decission making regarding adoption of a new technology, compared to the innovators, thus seeking to adopt a somewhat mature technology. Their judicious approach often position early adopters in central communication positions, as opinion makers \cite{rogers_model}.
Early adopters make up about 13.5\% of the consumers \cite{diffusion2} \\
\hline

Early majority &  
This group, representing 34 \% of consumers \cite{diffusion2} are the careful consumers, avoiding the big risks, by relying on feedback from the early adopter group. 
This group seldom represents the opinion makers, but rely on feedback and recommendations from others. Their more awaiting approach to adoption results in a somewhat later time for adoption of emerging innovations\cite{rogers_model}.\\
\hline

Late majority & 
The late majority also represents app. 34\% of the consumers\cite{diffusion2}, but with a somewhat more skeptical approach. Here, only innovations that has became more common items are adopted. The late majority group is characterized, in general, by having below average social status, and only very little opinion leadership\cite{rogers_model}.\\
\hline

Laggards & 
Laggards are the remaining 16\% of customers trying to avoid changes and having to adopt new innovations. This group may not adopt a new innovation until traditional alternatives are no longer available  cs\cite{diffusion2}. Focus is very much on traditions, the group tends to have the highest age among adopters, lowest financial fluidity, and primarily only being in contact with close friends and family. Thus this group does not represent any opinion leadership\cite{rogers_model} \\ 
\hline

\end{tabular}

\label{table:nonlin} % is used to refer this table in the text

\end{table}



?geels 


Applying adoption theory

Perspectives on current efforts

asses roberts in terms of technology lifetime vs risk vs invenstment


Describe technology adoption cycles eg. early adopters, laggards etc.

Common for the following models are, that they are used to describe the factors influencing decision makers in the adoptions of innovations.
of innovations.



\textbf{summary:}

Describe and argue the usage of Kaplan's modified rogers model
\textit{motivation ,experience, familiarity, context, knowledge, interest}

linear regeression ?

relate / score existing efforts in a matrix

\subsection{hypothesis}
\textit{"Visualizing energy consumption in realtime, by means of IT aided solution, and focusing onpotential savings, will enforce greener habits among the energy consumers"}

\textit{"By presenting a clear distinction between leisure and nessecery usage, the consumer groups, capable of adopting this technology, will be able to to adopt a greener mindset"}

\section{model}
By using Rogers models for technology diffusion and adoption, a framework will be established to asses the impact of energy saving campaigns the in the ?5 year? period 2006-2011.
The perspective applied is from an approach where the interest is to focus on IT aided solutions used to stimulate new greener habits among consumers.

Since the emerge of new IT aided technologies for communicating,  browsing the web, collecting data and communicating via social medias has experienced a massive growth the last decade, it will be a valid assumption that these technologies will play a central role in most Danish energy consumers everyday life.

This field, in particular the emerge of smartphones, combined with collective sensing and data-sharing presents a massive potential in terms of ability to collect, share and analyse data. Much of this relies on the continuous development and emerge of new technologies, thus models for adoption of new technology, in particular \textit{Everet Rogers theory behind Diffusion of innovations will be used}.

Existing evaluation of present conducted campaigns and where possible evaluation of current activities, both with primary focus on electricity usage, will be analyzed.


\section{current and past initiatives}
Below the most significant, and regional initittives, in Denmark across the last ?? years are outlined

\subsection{Energymarking of electric products}
ABCDEF

\subsection{Go energi}

\subsection{something more}


\subsection{Assesing inititatives}
Below the different initiatives are assessed, and rated among each other

%table below
\begin{table}[t2]
\caption{Assessing different initiatives} % title of Table
\centering  % used for centering table
\begin{tabular}{|p{1,5cm} |p{3.5cm} |p{2.5cm}|} % centered columns (4 columns)
\hline\hline                        %inserts double horizontal lines
Initiative & Charachteristics & adopter group \\ [0.5ex] % inserts table 
%heading
\hline                  % inserts single horizontal line
Go' energi & asdknkn & Early majority Late Majority  \\
\hline
\end{tabular}
\label{table:nonlin} % is used to refer this table in the text
\end{table}





\section{Stimulating awareness}

\subsection{Motivating usage}
Subsection text here.

\subsection{Visualization}
Subsection text here.

trendy thermostat ex..

suggest model for enforcing greener habits
\section{Conclusion and further work}


% use section* for acknowledgement
\section*{Acknowledgment}


The authors would like to thank...





% trigger a \newpage just before the given reference
% number - used to balance the columns on the last page
% adjust value as needed - may need to be readjusted if
% the document is modified later
%\IEEEtriggeratref{8}
% The "triggered" command can be changed if desired:
%\IEEEtriggercmd{\enlargethispage{-5in}}

% references section

% can use a bibliography generated by BibTeX as a .bbl file
% BibTeX documentation can be easily obtained at:
% http://www.ctan.org/tex-archive/biblio/bibtex/contrib/doc/
% The IEEEtran BibTeX style support page is at:
% http://www.michaelshell.org/tex/ieeetran/bibtex/
%\bibliographystyle{IEEEtran}
% argument is your BibTeX string definitions and bibliography database(s)
%\bibliography{IEEEabrv,../bib/paper}
%
% <OR> manually copy in the resultant .bbl file
% set second argument of \begin to the number of references
% (used to reserve space for the reference number labels box)

\begin{thebibliography}{1}

\bibitem{passive_to_active}
A.W. Kaplan, \emph{From passive to active about solar electricity: innovation decision
process and photovoltaic interest generation}, \relax Environmental Studies Program, Denison University, Granville, OH 43023, USA, July 1998.

\bibitem{udv_elpriser}
M. Togeby, A. E. Larsen, \emph{Udviklingen af elpriserne}, \relax Ea Energianalyse a/s Copenhagen Denmark, Feb.2011.

\bibitem{energistat}
Energistyrelsen, \emph{Energistatistik 2010}, \relax Energistyrelsen, Copenhagen Denmark, Sept.2011.

\bibitem{energipolitik_2020}
\emph{Aftale mellem regeringen (Socialdemokraterne, Det Radikale Venstre, Socialistisk Folkeparti) og Venstre, Dansk Folkeparti, Enhedslisten og Det Konservative Folkeparti
om den danske energipolitik 2012-2020} \relax Folketinget March 2012

\bibitem{rogers_model}
Rogers, E.M., \emph{Diffusion of Innovations}, \relax 4th ed. Free Press, New York. 1995

\bibitem{diffusion2} 
acb \emph{product diffusion curve},\relax www.quickmba.com/markting/product/diffusion

\bibitem{gilbert} 
Gilbert, M., Cordey-Hayes, M., 1996. \emph{Understanding the process of
knowledge transfer to achieve successful technological innovation}, \relax Technovation 1996

\end{thebibliography}




% that's all folks
\end{document}


